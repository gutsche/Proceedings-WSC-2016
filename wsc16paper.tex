%**************************************************************************
%*
%*  Paper: ``INSTRUCTIONS FOR AUTHORS OF LATEX DOCUMENTS''
%*
%*  Publication: 2016 Winter Simulation Conference Author Kit
%*
%*  Filename: wsc16paper.tex
%*
%*  Date: January 31, 2001   Time:  9:45 PM
%*      BASE of current version: Feb 01, 2010 (primary WSC'10 LaTeX file)
%*
%*  Word Processing System: TeXnicCenter and MiKTeX
%*
%*
%*  All files need the following
\input{wsc16style.tex}     % download from author kit.  Style files for wsc formatting. Don't remove this line - required for generating the final paper!

\documentclass{wscpaperproc}
\usepackage{latexsym}
%\usepackage{caption}
\usepackage{graphicx}
\usepackage{mathptmx}

%
%****************************************************************************
% AUTHOR: You may want to use some of these packages. (Optional)
\usepackage{amsmath}
\usepackage{amsfonts}
\usepackage{amssymb}
\usepackage{amsbsy}
\usepackage{amsthm}
%****************************************************************************



%
%****************************************************************************
% AUTHOR: If you do not wish to use hyperlinks, then just comment
% out the hyperref usepackage commands below.

%% This version of the command is used if you use pdflatex. In this case you
%% cannot use ps or eps files for graphics, but pdf, jpeg, png etc are fine.

\usepackage[pdftex,colorlinks=true,urlcolor=blue,citecolor=black,anchorcolor=black,linkcolor=black]{hyperref}

%% The next versions of the hyperref command are used if you adopt the
%% outdated latex-dvips-ps2pdf route in generating your pdf file. In
%% this case you can use ps or eps files for graphics, but not pdf, jpeg, png etc.
%% However, the final pdf file should embed all fonts required which means that you have to use file
%% formats which can embed fonts. Please note that the final PDF file will not be generated on your computer!
%% If you are using WinEdt or PCTeX, then use the following. If you are using
%% Y&Y TeX then replace "dvips" with "dvipsone"

%%\usepackage[dvips,colorlinks=true,urlcolor=blue,citecolor=black,%
%% anchorcolor=black,linkcolor=black]{hyperref}
%****************************************************************************



		



%
%****************************************************************************
%*
%* AUTHOR: YOUR CALL!  Document-specific macros can come here.
%*
%****************************************************************************

% If you use theoremes
\newtheoremstyle{wsc}% hnamei
{3pt}% hSpace abovei
{3pt}% hSpace belowi
{}% hBody fonti
{}% hIndent amounti1
{\bf}% hTheorem head fontbf
{}% hPunctuation after theorem headi
{.5em}% hSpace after theorem headi2
{}% hTheorem head spec (can be left empty, meaning `normal')i

\theoremstyle{wsc}
\newtheorem{theorem}{Theorem}
\renewcommand{\thetheorem}{ \arabic{theorem}}
\newtheorem{corollary}[theorem]{Corollary}
\renewcommand{\thecorollary}{\arabic{corollary}}
\newtheorem{definition}{Definition}
\renewcommand{\thedefinition}{\arabic{definition}}


%#########################################################
%*
%*  The Document.
%*
\begin{document}

%***************************************************************************
% AUTHOR: AUTHOR NAMES GO HERE
% FORMAT AUTHORS NAMES Like: Author1, Author2 and Author3 (last names)
%
%		You need to change the author listing below!
%               Please list ALL authors using last name only, separate by a comma except
%               for the last author, separate with "and"
%
\WSCpagesetup{Gutsche}

% AUTHOR: Enter the title, all letters in upper case
\title{Dark Matter And Super Symmetry: Exploring And Explaining The Universe With Simulations At The LHC}

% AUTHOR: Enter the authors of the article, see end of the example document for further examples
\author{Oliver Gutsche\\ [12pt]
Scientific Computing Division \\
Fermi National Accelerator Laboratory\\
P.O.Box 500\\
Batavia, IL, 60510, USA\\
}

\maketitle

\section*{ABSTRACT}
The Large Hadron Collider (LHC) at CERN in Geneva, Switzerland, is one of the largest machines on this planet. It is built to smash protons into each other at unprecedented energies to reveal the fundamental constituents of our universe. The 4 detectors at the LHC record multi-Petabyte datasets every year. The scientific analysis of this data requires equally large simulation datasets of the collisions based on the theory of particle physics, the Standard Model. The goal is to verify the validity of the Standard Model or of theories that extend the Model like the concepts of Super Symmetry and an explanation of Dark Matter. I will give an overview of the nature of simulations needed to discover new particles like the Higgs Boson in 2012, and review the different areas where simulations are indispensable: from the actual recording of the collisions to the extraction of scientific results to the conceptual design of improvements to the LHC and its experiments.

\section{INTRODUCTION}
\label{sec:intro}
High Energy Physics (HEP) strives to develop a detailed mathematical understanding of nature at the smallest elementary level. It’s science is based on the interplay between the theory framework that describes elementary particles and elementary forces between them; and the experimental detection of particles and measurements of their interactions. It calls for probing nature at ever increasing detail to unlock the last mysteries of our universe.

The theory of particle physics is called the Standard Model and describes the universe through 12 particles and their anti-particles, and 4 fundamental forces represented by their own force particles. Particles are called fermions and have half-integer spins (one of fundamental properties or quantum numbers of particles) and respect the Pauli exclusion principle (not two particles can be identical in all their quantum numbers). There are 12 fermions, separated into 6 leptons (electron, electron neutrino, muon, muon neutrino, tau, tau neutrino), and 6 quarks (up, down, charm, strange, bottom, top). The hydrogen atom consists of an electron orbiting a proton, which consist of 2 up and 1 down quark.

Force particles are bosons with integer spin and describe the fundamental forces: the electro-magnetic force represented by the photon, the weak force represented by the W and Z bosons, the strong force represented by the gluon, and gravitation which is the least known fundamental force and believed to be represented by the graviton, but this is not yet proven.

The Standard Model uses the principles of Quantum Field Theory to mathematically describe particles and their interactions. Today, the Standard Model is very successful in describing matter and their interactions. It took many years and numerous experiments to develop and veify the theory. Although it is therotically self-consistent, it cannot describe gravitation, account for the accelerating expansion of the universe, and cannot incorporate that neutrinos have mass, as proved through the detection of neutrino oscillations. Therefore the field of particle physics is very active in understanding these topics and improve and enhance the Standard Model. The goal is to develop an all-encompassing theory.

\section{SIMULATION}
\label{sec:simulation}

The rules of particle physics are governed by quantum physics. Therefore description and predictions of particle physics interactions and phenomena have to deal with probablities and sufficiently large statistics to make meaningful statements. For the experiment, this means that only particle interactions that have been observed in sufficient numbers can be compared to theoretical predictions. Fortunately, experiments are mostly event driven. Individual particle interactions, may they be collisions or otherwise produced, can be treated individually. This makes particle physics a highly parallelizable science.

\subsection{EVENT SIMULATION} 
\label{subsec:eventsimulation}

The empirical calculation of a particle interaction is only possible in approximation and is called event simulation. The problem of describing mathematically a simple collision of two particles producing a different set of two particles at the first and most basic order can be calculated simply by calculating the exchange of a force particle. First one calculates the interaction of the two incoming particles with the force particle (first vertex) and then the interaction of the same force particle with the outgoing second set of two particles (second vertex). But higher order corrections can play a significant role in calculating this collision. There are two kinds of higher order corrections that are important. 

The first is extending the number of particles and force particles that are produced and exchanged between the first and second vertex. In particle physics speak, the number of vertices is increased and therefore the order of the calculation increases. The more orders are calculated, the closer the approximation of the calculation to the truth. The first order is called the "leading order (LO)" and the next following orders are called "next-to-leader order (NLO)" and "next-to-next-to-leading order (NNLO)". NNLO calculations are currently state-of-the-art in particle physics.

The second kind of higher order correction is harder to approximate. These corrections have to do with additional force particles radiated from the incoming and outgoing particles and therefore changing the calculation from above. As there are many possibilities to radiate extra particles with a wide range of possible kinematical parameters (energy, momentum to name a few), a precise or even mathematical approximation is very difficult. Particle physics uses Monte Carlo techniques to tackle these problems. A sufficiently large number of interactions is calculated starting from a randum number seed. Every simulated interaction is using a different randum number seed. The random number is used to choose the composition of the empirical calculation up to the order specified, meaning it chooses randomly how many vertices need to be calculated and what the internal kinematical parameters of these vertices are. Then the random number is used to determine if and how particles are radiated and this is included in the calculation as well. Probability density functions are used to constrain the possibilities that can be chosen. These functions are determined by theory confirmed by experiment or by experiment directly.

The sufficiently large statistics of these simulated interactions allows to make statistically significant statements about the interaction of interest, including kinematic distributions of the final state, through averaging the results. Sufficiently large statistics means that we are covering the phase space of allowed configurations sufficiently that on average we get the correct result.

\subsection{DETECTOR SIMULATION} 
\label{subsec:detectorsimulation}

To verify or even extend the theory, experiments and the comparison of experimental results with theoretical predictions are needed. The event simulation step described in Sec.~\ref{subsec:eventsimulation} is not directly comparable with a potential experimental setup. This is because we can only very few of the particles of the Standard Model produce, control and measure. 

Of the leptons, only the electron and muon are stable and observable. The tau is decaying into electrons or muons very quickly and the neutrinos are so weakly interacting that neutrino physics is its own branch in particle physics. The quarks are not observable individually at all. Governed by the strong force and a concept called confinement, quarks can only be observed in 2 or 3 quark bound states. 3 quark bound states are called baryons and the most prominent examples are protons and neutrons. 2 quark bound states are called mesons. Because we also cannot control the initial state of a particle interaction to all extend. We can collide for example protons with protons, which means two 3-quark states are colliding, giving multiple possibilities how a certain quark quark ineraction can be produced. Also these effects can be simulated using Monte Carlo techniques.

The same effect that binds quarks in mesons and baryons also governs the constitution of the final state of an interaction. Because individual quarks cannot exist on their own, quarks that are produced in an interaction hadronize by collecting additional quarks from the vaccum (spontaneous quark anti-quark production following $E=mc~2$) or from neigbouring quarks in the final state of the interaction. Hadronization is described as well with the help of Monte Carlo techniques using underlying models and theories. The same is true for particles that fragment or decay while transversing a detector material.

Detectors are made of matter and detect particles by interacting with them. Dtectors can consists of a gas that is ionized by particles, or a plastic that produces light when a particle transverses the material, or a semi-conductor in which a current can be measured when a particle is passing through. In simulation, we know the final state of an interaction. We encapsulated all theories and models to describe the enegry depositions of particles in material in a single package that is generally used, also outside particle physics, called Geant. Geant simulates the energy deposition of a particle in material, depending on the type of material, the amount of material and the three-dimensional cmoposition of various materials that make up a detector or detector system. 

The last part of the detector simulation step is to translate the energy depositions of the particles simulated by Geant into electrical signals of the used detectors and detector systems. After this step, a simulated interaction and an experimentally measured interaction are equivalent can be treated the same in the following.






















Also called elementary particle physics, its experimental results are based on the analysis of many individual detector measurements in comparison to corresponding simulations that are based on the current understanding of the theory. Because of this, HEP was and is traditionally a very data intensive and trivially parallelizable science discipline.

The LHC~\cite{1748-0221-3-08-S08001}

\section*{ACKNOWLEDGMENTS}
Place the acknowledgments section, if needed, after the main text, but before any appendices and the references. The section heading is not numbered.
These instructions are adapted from instructions that have been updated and improved by proceedings editors and several other individuals, who are too numerous to name separately (our apologies, but it is necessary), since the first set of instructions were written by Barry Nelson for the 1991 WSC.

\appendix


% Please don't exchange the bibliographystyle style
\bibliographystyle{wsc}
% AUTHOR: Include your bib file here
\bibliography{wsc16paper}

\section*{AUTHOR BIOGRAPHIES}

\noindent {\bf Oliver Gutsche} is a staff scientist at the Fermi National Accelerator Laboratory and member of the CMS collaboration of 2,500 physicists, which is operating one of the 4 detectors at the Large Hadron Collider (LHC) at CERN in Geneva, Switzerland. After the Higgs Boson discovery in 2012, his research is focusing on new physics beyond the established theory of particle physics called the Standard Model, especially in the areas of Super Symmetry and Dark Matter. In his role as Assistant Head of the Scientific Computing Division, Dr. Gutsche coordinates the computing needs of the High Energy, Neutrino and Muon Particle Physics experiments at the laboratory. He has intimate knowledge of the large scale computing solutions used for the LHC experiments to analyze multi-Petabyte size datasets on distributed computing infrastructures of many 100,000 cores, having architected many of the used systems and leading the computing operations team of CMS during the first running period of the LHC. His email address is \email{gutsche@fnal.gov}.\\


\end{document}

